\section{Results \label{sec:quantresults}}

Table \ref{table:data_simulations} reports moments in the simulations of the model for the baseline parametrization. We base statistics on pre-default simulation samples, except for the computation of default frequencies which are computed on the entire sample. We simulate the model for a number of periods that allows us to extract 1000 samples of 35 consecutive periods before a default. We focus on samples of 35 periods because we compare the artificial data generated by the model with Argentine data from the first quarter of 1993 to the last quarter of 2001.\footnote{The qualitative features of this data are also observed in other sample periods and in other emerging markets (see, for example, \citealp{AG_cycles}, \citealp{Alvarez}, \citealp{Bora}, \citealp{Neumeyer05}, and \citealp{drives}).} In order to facilitate the comparison of simulation results with the data, we only consider simulation sample paths for which the last default was declared at least four periods before the beginning of each sample. Default frequencies are computed using all simulation data.

Table \ref{table:data_simulations} shows the simulation results of the benchmark model (only noncontingent debt) under the parametrization described above. It also shows how the simulation results change if we consider rational expectations lenders as in \cite{Chatty} (i.e., if we set $\theta=0$). Overall, the table shows that our benchmark calibration with robust lenders is successful in accounting for the moments in the data. As in the data, in the simulations of the baseline model, consumption and income are highly correlated and the spread is countercyclical (non-targeted moments). Consumption is more volatile than income, which is consistent with the findings in \cite{Neumeyer05} and \cite{AG_cycles}. The benchmark calibration closely matches the targeted moments as well. We note that the calibration with rational expectations lenders also approximates moments in the data reasonably well. The crucial difference is that only with robust lenders is the model able to match simultaneosuly the moments of the spread and default frequency in the data (3 percent). The model with rational expectations lenders is able to match the mean spread level but at the expense of a much larger default probability (5.5 percent). \cite{PouzoPresno2016} show that the model with robust lenders is also able to match other quantiles of the spread in the data.

\newcolumntype{g}{>{\bfseries\centering\arraybackslash}m{0.89in}}
\begin{table}[!hbtp]\centering \small
\caption{Data and model simulations.}\label{table:data_simulations}
\begin{tabular}{@{}lccc@{}} \toprule
 & \textbf{Data} & \multicolumn{1}{c}{\textbf{Benchmark}} & \multicolumn{1}{g}{\textbf{Rational Expectations}} \\\midrule
Spread                   & 8.15            & 8.15            & 8.1            \\
Std Spread               & 4.58            & 4.6            & 4.5            \\
Debt                     & 46          & 44           & 48.7           \\
Std(c)/Std(y)            & 0.87            & 1.25           & 1.24           \\
Corr(y,c)                & 0.97           & 0.98           & 0.98           \\
Corr(y,tb/y)             & -0.77          & -0.68           & -0.71          \\
Corr(y,spread)           & -0.72        & -0.76           & -0.77          \\
Default Prob             & 3.0            & 3.0            & 5.5            \\
DEP                      & -              & 31\%           & -           \\
\bottomrule
\end{tabular}
\end{table}

Table \ref{table:data_simulations} reports an additional statistic which is not standard in the literature of sovereign debt: the detection error probabilities (DEP). While we calibrate the lenders' desire for robustness as discussed in the previous section, we also relate the value of $\theta$ to detection error probabilities. DEP is the probability that, given data simulated from one model, the other model’s likelihood function is larger (with equal weight on which model generated the data). These are commonly used in the robust control literature to assess the plausibility of the amount of probability distortions allowed in the economy. The DEP capture the probability with which an agent, with a limited amount of data, is able to distinguish between the worst-case and the benchmark densities. If the DEP is 0, the models are so different that the agent can perfectly differentiate them. In contrast, when the DEP is 0.5 the models are identical and the agent is unable to distinguish them. A high DEP therefore suggests that the misspecification implicit in the distorted model is plausible and validates the agent desire for robustness. In the robust control literature, a calibration is deemed to imply an acceptable degree of aversion to model uncertainty when the assoaciated DEP is of at least 20\% (see \citealp{BarillasHansenSargent}). In our case, the DEP is 31\% which implies that we allow for reasonable amounts of probability distorions in the economy.\footnote{We follow \citet{BarillasHansenSargent} and compute model detection error probabilities as the probability of misclassifying a 240-period sample generated by the approximating model (after a 2000-period burn-in to avoid dependence on initial conditions) as coming from the worst-case model, over 2000 repetitions. As \citet{PouzoPresno2016} note, much larger DEP would result if we used the length of the calibration sample.}

Next, we turn to the analysis of allowing the government to issue state-contingent debt. In table \ref{table:PP}, the column ``Noncontingent'' shows the simulations results discussed above and the column ``Threshold'' shows the simulation results when the government can issue an income-indexed bond with parameters $\tau = \bar{y}$ and $\alpha = 1$. This bond structure in the model intends to capture the GDP-linked bond that Argentina issued in 2005 (the details of this bond are discussed in Section \ref{sec:2permodel}). Table \ref{table:PP} shows that the qualitative results from the two-period model carry over into the quantitative model. This income-indexed bond generates substantial welfare gains if the small open economy faces rational-expectations lenders. As the government does not need to make coupon payments in any income realization lower to the mean, the default probability and, thus, the spread are almost eliminated. This allows the government to increase its indebtedness and reduce the volatility of consumption. The threshold bond effectively expands the government borrowing opportunities, allowing for larger indebtedness at more favorable prices. This result under rational expectations corresponds to the puzzle in the literature. The structure of the threshold bond provides welfare gains by improving the allocation of risk among the government and its creditors.
  
\begin{table}[!hbtp]\centering\small
\caption{Statistics for different debt structures.}  \label{table:PP}
\begin{tabular}{@{}lccc@{}}\toprule
  & \multicolumn{3}{c}{Rational Expectations} \\\cmidrule{2-4}
\textbf{Statistic} & \textbf{Noncontingent} & \textbf{Threshold} & $\alpha = 1$\\\midrule
Spread                   & 8.1            & 0.36            & 7.2           \\
Std Spread               & 4.5            & 0.23            & 3.7            \\
Debt                     & 48.7           & 116.5           & 50.8           \\
Std(c)/Std(y)            & 1.24           & 0.82           & 1.22           \\
Corr(y,c)                & 0.98           & 0.55           & 0.98           \\
Corr(y,tb/y)             & -0.71          & 0.54           & -0.67          \\
Corr(y,spread)           & -0.77          & -0.87           & -0.79          \\
Default Prob             & 5.5            & 0.3            & 5.3            \\
Welfare Gains            & -              & 1.19           & 0.09           \\
DEP                      &    -           &    -           &    -          \\
  \bottomrule
\end{tabular}%
\begin{tabular}{@{}lccc@{}}\toprule
& \multicolumn{3}{c}{$\theta = 1.6155$ (benchmark)} \\\cmidrule{2-4}
& \textbf{Noncontingent} & \textbf{Threshold} & $\alpha = 1$ \\\midrule
& 8.15           & 11.1           & 7.1           \\
& 4.6            & 1.58            & 3.6            \\
& 44.0           & 67.6           & 46.1           \\
& 1.25           & 0.84           & 1.23           \\
& 0.98           & 0.93           & 0.98           \\
& -0.68          & 0.52           & -0.64          \\
& -0.76          & -0.63          & -0.77          \\
& 3.0            & 0.0            & 2.6            \\
& -              & -0.37          & 0.07           \\
& 31\%           & 20\%           & 30\%          \\ 
\bottomrule 
\end{tabular}
\begin{tablenotes1}
  Threshold debt pays if income is above the mean and payments are linearly indexed with $\alpha = 1$.
\end{tablenotes1}
\end{table}
  
However, when it faces robust lenders the government is made worse off by the introduction of the threshold bond. In this case, while the threshold bond still reduces the volatility of consumption relative to income and eliminates most of the default risk, it still leads to a large increase in bond spreads. The sovereign spread increases from 8.15 to 11.1 percent when lenders are robust. As in the discussion in Section \ref{sec:2permodel}, in this case sovereign spreads not only reflect the default probability but also include amiguity premia from default and from stipulated contingency. Given that the government never defaults with this bond structure, the default premium and the ambiguity premium associated with default are both zero. But robust lenders' probability distortions amplify the likelihood of states in which the bond promises no repayment. This leads to a large ambiguity premium related to the contingency of the bond which explains the resulting levels of the spread and, ultimately, the associated welfare losses.

Finally, the change in the bond structure affects the stochastic discount factor used by lenders and the probability distortions they use. Hence, as we move from one type of bond to another, we are not only changing the debt instrument but also potentially affecting the total amount of probability distortion allowed in the economy. That means that lenders in one of the economies could have more pessimistic beliefs than in the other economy, which per se could have strong implications for the pricing of the bonds. Thus, we recompute the DEP to make sure that the worst-case model used by lenders does not become so different from the approximating model that limited data can tell them apart. We find that DEP remain over the 20\% threshold advocated by \cite{BarillasHansenSargent} when the sovereign issues the ``threshold'' bond, without the need to recalibrate the benchmark value of $\theta$.\footnote{We could also recalibrate $\theta$ to ensure a constant DEP; however, since the difference in DEP is moderate, we believe that recalibrating $\theta$ would not change our main results.}

Consistent with the results in the two-period model, these findings show that, through the lens of our model, the unexplained portion of the spread of the Argentinian GDP-linked bond \citep*[which the literature has labeled as a novelty premium, see][]{ChamonCostaRicci2008} is in fact an ambiguity premium. Moreover, the welfare losses associated with this bond structure could also rationalize why countries have not issued these bonds more frequently in practice.

We now turn to the optimal state-contingent bond design. For each calibration, we maximize the welfare of the sovereign by choosing the parameter values $\tau$ and $\alpha$. Table \ref{table:optimal_structure} shows the simulation results. We find that the optimal bond design depends on the type of lenders the government is facing. In particular, while the threshold level $\tau$ only covers the very left tail of the distribution in both cases, the optimal degree of indexation $\alpha$ is lower when lenders feature preferences for robustness. In all cases, the optimal state-contingent bond substantialy reduces both default risk and the volatility of consumption. At the same time, this allows the government to increase its indebtedness. When lenders have rational expectations, the decline in default risk implies a similar reduction in the sovereign spread. On the other hand, when lenders are robust sovereign spreads are an order of magnitude higher than default risk. As discussed in Section \ref{sec:2permodel}, this spread level reflects ambiguity premia related to the contingency of the bond. Overall, choosing the optimal state-contingent bond design results in large welfare gains, although these are larger with rational expectations lenders.\footnote{We also find that for a given state-contingent structure, welfare gains are decreasing in the robustness parameter $\theta$.}



\begin{table}[!hbtp]\centering\small 
  \caption{Statistics under the optimal state-contingent bond with and without robust lenders.} \label{table:optimal_structure}
  \begin{tabular}{@{}lccc@{}}\toprule
    %& \multicolumn{2}{c}{\citet{PouzoPresno2016}} \\\cmidrule{2-3}
  \textbf{Statistic}  & \textbf{Rational Expectations} & \textbf{Benchmark}  \\
  & $\tau$ = 0.875, $\alpha$ = 7 & $\tau$ = 0.875, $\alpha$ = 5 &\\\midrule
  Spread &   0.1          &       2.8        \\
  Std Spread &   0.04          &         0.13            \\
  Debt &   79.3         &           65.9           \\
  Std(c)/Std(y) &   0.76         &           0.96          \\
  Corr(y,c) &   0.99        &           0.98        \\
  Corr(y,tb/y) &   0.98        &           0.25      \\
  Corr(y,spread) &   -0.91        &          -0.67       \\
  Default prob &  0.1             &    0.23                  \\
  Welfare gains & 1.79              & 0.79                 \\
  DEP & -              & 25\%                 \\
    \bottomrule 
  \end{tabular}
  \end{table}
  

\subsection{Sensitivity Analysis \label{subsec:sensitivity}}

In this section, we conduct a sensitivity analysis to illustrate that the quantitative importance of the mechanism proposed in this paper is robust to the calibration of the model. 

\subsubsection{The level of debt in the data}

Our benchmark calibration targets an average debt level of 46\% of GDP, which corresponds in the data to the amount of external government debt held by non-residents. However, previous studies that also use Argentina as a case study include external debt held by residents as well, which implies targeting an average debt to quarterly GDP ratio of 100\%. Following \cite{Chatty}, we scale this target by 0.7 in order to account for the recovery rate on defaulted debt by Argentina. Thus, we recalibrate our model with noncontingent debt to match an average debt level ot 70\% of GDP and the same remaining targets from our benchmark calibration.

Table \ref{table:CE} reports moments computed from simulated data for this alternative parametrization. As in the main exercise, we present results for both rational expectations and robust lenders (i.e., with $\theta=0$ but maintaining all other parameters unaltered). The table shows that the quantitative results from our benchmark calibration carry over to this alternative calibration. The ``threshold'' bond eliminates default risk and, in the case of rational-expectations lenders, reduces the spread and generates substantial welfare gains. In contrast, robust lenders charge a meaningful ambiguity premium to this instrument. This results in welfare losses for the government.

\begin{table}[!hbtp]\centering\small 
\caption{Statistics based on the alternative calibration targeting a debt level of 70\% of GDP.} \label{table:CE}
\begin{tabular}{@{}lccc@{}}\toprule
  & \multicolumn{3}{c}{Rational Expectations} \\\cmidrule{2-4}
\textbf{Statistic} & \textbf{Noncontingent} & \textbf{Threshold} & $\alpha = 1$\\\midrule
Spread                   & 8.5            & 0.6            & 6.8            \\
Std Spread               & 4.3            & 0.4            & 3.0            \\
Debt                     & 69.9           & 159.6           & 74.4           \\
Std(c)/Std(y)            & 1.24            & 0.83           & 1.21           \\
Corr(y,c)                & 0.98           & 0.53           & 0.98           \\
Corr(y,tb/y)             & -0.7          & 0.52           & -0.62          \\
Corr(y,spread)           & -0.77        & -0.87           & -0.78          \\
Default Prob             & 5.8            & 0.56            & 5.3            \\
Welfare Gains            & -              & 1.86           & 0.27           \\
DEP				      & - 			& - 			& -                  \\
  \bottomrule
\end{tabular}%
\begin{tabular}{@{}lccc@{}}\toprule
  & \multicolumn{3}{c}{$\theta = 1.6155$} \\\cmidrule{2-4}
  & \textbf{Noncontingent} & \textbf{Threshold} & $\alpha = 1$ \\\midrule
& 8.4            & 15.5           & 7.1            \\
& 4.4            & 2.3            & 3.1            \\
& 62.6           & 87.7           & 67.2           \\
& 1.25           & 0.82           & 1.22           \\
& 0.98           & 0.94           & 0.98           \\
& -0.67          & 0.58           & -0.6          \\
& -0.75          & -0.61          & -0.77           \\
& 2.3            & 0.12            & 1.8            \\
& -              & -0.87          & 0.2           \\
& 24\%         & 14\%             & 22\%             \\
\bottomrule
\end{tabular}
\begin{tablenotes1}
  Threshold debt pays if income is above the mean and payments are linearly indexed with $\alpha = 1$.
\end{tablenotes1}
\end{table}

Finally, Table \ref{table:CE2} reports moments computed from simulated data for this alternative parametrization under the optimal state-contingent debt structure. Again, the quantitative results from our benchmark calibration carry over to this alternative calibration. The optimal bond design features a similar threshold level $\tau$ at the very left tail of the distribution both under rational expectations and robust lenders, while the optimal degree of indexation $\alpha$ is lower when facing robust lenders.

\begin{table}[!hbtp]\centering\small 
\caption{Statistics based on the alternative parametrization under the optimal state-contingent bond.} \label{table:CE2}
\begin{tabular}{@{}lccc@{}}\toprule
 % & \multicolumn{2}{c}{\citet{Chatty}} \\\cmidrule{2-3}
\textbf{Statistic} & \textbf{Rational Expectations} & \textbf{Robustness}\\
& $\tau$ = 0.75, $\alpha$ = 4 & $\tau$ = 0.8, $\alpha$ = 3 &\\\midrule
Spread                   &   0.02          & 2.83                        \\
Std Spread               &   0.02          & 0.11                     \\
Debt                     &    119.8        & 95.7                 \\
Std(c)/Std(y)            &     0.8        & 0.99                   \\
Corr(y,c)                &     0.99       & 0.98                   \\
Corr(y,tb/y)             &     0.98      & 0.13                  \\
Corr(y,spread)           &   -0.42         & -0.17                      \\
Default Prob             &  0.04           & 0.17                      \\
Welfare Gains            & 3.2             & 1.44                    \\
DEP                         & -               &   20\%                      \\
  \bottomrule
\end{tabular}%

\end{table}


%\subsubsection{The degree of lenders' robustness}

\subsubsection{The average spread and spread volatility}

While using Argentina as a case study is commonly done in the literature of sovereign default, the spread is much higher and volatile than in other economies with sovereign default risk. In this subsection we analyze whether our findings depend on the characteristics of the economy (e.g., average spread and spread volatility). To this end, we study two alternative calibrations by modifying the output cost parameters from our benchmark calibration. Table \ref{table:simulations_sensitivity} reports moments computed from simulated data when we only change either $d_1$ (from 0.296 to 0.31) or $d_0$ (from -0.255 to -0.235) from the benchmark calibration. When the government issues noncontingent debt, these alternative calibrations feature both lower and less volatile spreads than our baseline calibration, and still allow for plausible amounts of model uncertainty as implied by the DEP above the 20\% threshold.

Our findings from our benchmark calibration hold as well for these alternative calibrations. Table \ref{table:simulations_sensitivity} shows that under both alternative calibrations the government is worse off when issuing the `'threshold'' bond due to the elevated ambiguity premia charged by robust lenders (despite the negligible default probabilities). Finally, Table \ref{table:simulations_sensitivity2} reports moments computed from simulated data for these alternative parametrizations under their optimal state-contingent debt structures. Again, in both cases the optimal debt design avoids promising likely states without repayments by choosing the a low threshold level $\tau$, and also avoids large swings in repayments across the state space by setting gentle slopes (relatively low values for $\alpha$). Overall, these results confirm the findings using the benchmark calibration.


\begin{table}[!hbtp]\centering\small 
\caption{Statistics based on the alternative calibrations varying the output cost parameters.} \label{table:simulations_sensitivity}
\begin{tabular}{@{}lccc@{}}\toprule
  & \multicolumn{3}{c}{$d_1=0.31$} \\\cmidrule{2-4}
\textbf{Statistic} & \textbf{Noncontingent} & \textbf{Threshold} & $\alpha = 1$\\\midrule
Spread                   & 5.78            & 11.29            & 4.99            \\
Std Spread               & 3.3            & 1.5            & 2.42            \\
Debt                     & 61           & 74           & 65           \\
Std(c)/Std(y)            & 1.33            & 0.79           & 1.29           \\
Corr(y,c)                & 0.97           & 0.92           & 0.97           \\
Corr(y,tb/y)             & -0.67          & 0.59           & -0.62          \\
Corr(y,spread)           & -0.75        & -0.45           & -0.74          \\
Default Prob             & 1.6            & 0.0            & 1.3            \\
Welfare Gains            & -              & -0.79           & 0.12           \\
DEP				      & 28\% 			& 19\% 			& 26\%                  \\
  \bottomrule
\end{tabular}%
\begin{tabular}{@{}lccc@{}}\toprule
  & \multicolumn{3}{c}{$d_0=-0.235$} \\\cmidrule{2-4}
  & \textbf{Noncontingent} & \textbf{Threshold} & $\alpha = 1$ \\\midrule
& 4.32            & 11.58           & 4.00            \\
& 2.5            & 1.4            & 1.8            \\
& 69           & 78           & 76           \\
& 1.32           & 0.76           & 1.30           \\
& 0.96           & 0.92           & 0.96           \\
& -0.65          & 0.67           & -0.57          \\
& -0.78          & -0.33          & -0.71           \\
& 0.9            & 0.0            & 0.8            \\
& -              & -1.11          & 0.12           \\
& 28\%         & 19\%             & 24\%             \\
\bottomrule
\end{tabular}
\begin{tablenotes1}
  Threshold debt pays if income is above the mean and payments are linearly indexed with $\alpha = 1$.
\end{tablenotes1}
\end{table}

\begin{table}[!hbtp]\centering\small 
\caption{Statistics based on the parametrizations varying the output cost parameters under the optimal state-contingent bond.} \label{table:simulations_sensitivity2}
\begin{tabular}{@{}lccc@{}}\toprule
 % & \multicolumn{2}{c}{\citet{Chatty}} \\\cmidrule{2-3}
\textbf{Statistic} & \textbf{$d_1=0.31$} & \textbf{$d_0=-0.235$}\\
& $\tau$ = 0.875, $\alpha$ = 3 & $\tau$ = 0.875, $\alpha$ = 2 &\\\midrule
Spread                   &   2.04          & 1.67                        \\
Std Spread               &   0.3          & 0.4                     \\
Debt                     &    89        & 99                 \\
Std(c)/Std(y)            &     1.03        & 1.05                   \\
Corr(y,c)                &     0.97       & 0.96                  \\
Corr(y,tb/y)             &     -0.02      & -0.09                  \\
Corr(y,spread)           &   -0.73         & -0.75                      \\
Default Prob             &  0.1          & 0.1                      \\
Welfare Gains            & 0.92             & 0.89                    \\
DEP                         & 25\%               &  26\%                      \\
  \bottomrule
\end{tabular}%

\end{table}




