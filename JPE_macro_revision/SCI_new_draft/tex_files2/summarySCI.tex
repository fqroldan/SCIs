\section{Recent experiences with state-contingent debt and the case for robustness \label{sec:summarySCI}}

State-contingent debt instruments are not used frequently in sovereign borrowing. Subsection \ref{subsec:summarySCI_1} summarizes some recent cases, along with a description of the contingency involved in the various issuances, and shows that these bonds have traded at a large premia. Subsection \ref{subsec:summarySCI_2} discusses why we consider that robustness provides a better rationale for this underpricing over alternative explanations in the literature.

\subsection{The design of state-contingent debt and the `novelty' premium \label{subsec:summarySCI_1}}

\citet*{databaseSCI} compiles 38 instances of issuances of sovereign state-contingent debt, ranging from cotton bonds issued by the Confederate States of America in 1863 to the IBRD Cat bonds issued in 2018 by Peru, Colombia, Chile, and Mexico to stipulate lower debt payments in case of earthquakes. The vast majority of bonds are structured in a way that promises reduced or no payments if some measure of income (such as output or a key export price, among others) falls below a certain threshold. 

\citet*{ChamonCostaRicci2008} describe in detail the GDP-warrants issued by Argentina as part of its 2005 debt restructuring. This contract was characterized by a threshold and a slope. Payments only occured if the state of the economy satisfied three conditions. First, real GDP growth must exceed `baseline' GDP growth in the reference year.\footnote{Baseline GDP growth was set by the authorities to gradually converge from an initial level of $4.3\%$ in 2005 to a long-run level of $3\%$ at the maturity of the bond in 2034.} Second, the level of real GDP had to be higher than the compounding of the baseline growth rates. Finally, payments only occured if the cumulative amount of past payments was below another threshold (of $48$ cents of the currency of denomination per unit of security). If the threshold was satisfied, the slope component of the bond meant that payments were a fraction of the difference between the actual and baseline levels of real GDP. \citet*{ChamonCostaRicci2008} use Monte Carlo simulations based on historical data to compute theoretical prices for the GDP warrants from 2005 to 2007. They find wide spreads: between 300 and 400bps which they attribute to the default risk of other securities, and a residual of 1200bps (which declined over time to about 600bps) which they interpret as a premium for `novelty.'

The decline in the residual premium featured in \citet*{ChamonCostaRicci2008} could indeed justify labeling it as `novelty' implying that these were relatively exotic assets and, thus, market participants needed time become familiar with the trading of these state-contingent bonds. Moreover, the severe underpricing could also be related to a poor economic outlook specific to Argentina, given its fragile macroeconomic framework. However, \cite{Taehoon} update the exercise in \citet*{ChamonCostaRicci2008} with more recent data (also extending to additional countries), and their results suggest otherwise. Figure \ref{Figures_Taehoon} shows the estimated spread decomposition for the GDP-warrant issued by Argentina, Greece and Ukraine. The residual premium, which they label as `SCDI Premium,' has remained at a high level in the case of Argentina over a period of 15 years. They also find a significant residual premium in the recent issuances by Greece and Ukraine. These results indicate that the underpricing of these instruments is not a country-specific issue and does not seem to be related to the novelty of the instrument.

\begin{figure}[!hbtp]
	\includegraphics[width=0.495\textwidth]{fig2_arg.eps}
	\includegraphics[width=0.495\textwidth]{fig2_grc.eps}
	\begin{center}
		\includegraphics[width=0.5\textwidth]{fig2_ukr.eps}
	\end{center}
	\caption{GDP-linked security premia. \label{Figures_Taehoon}}
	\begin{figurenotes}
	The figure shows the estimated spread decomposition in \cite{Taehoon} for the GDP-warrants issued by Argentina (top left), Greece (top right) and Ukraine (bottom). 
	\end{figurenotes}
\end{figure}

\subsection{Why robustness? \label{subsec:summarySCI_2}}

Among the existing arguments for the lack of indexation in sovereign debt markets discussed in the Introduction, concerns about data accuracy and moral hazard related to misreporting statistics seem to be the most shared in policy circles. However, these concerns should not be overemphasized. First, these arguments should also apply to inflation-linked bonds which many countries have managed to trade a reasonable prices.\footnote{Notice that inflation-linked bonds typically do not involve repayment thresholds.} Second, in the case of GDP-linked bonds, governments would pay significant political costs for underreporting growth. Moreover, long maturities could correct for the moral hazard problem, as the GDP growth rate would be less manipulable over the longer term. In fact, in the case of Argentina, the government has been criticized for allegedly \emph{over}reporting growth and underreporting inflation. In the case of Greece, the issuance of GDP-linked securities occurred under the scrutiny of the European Commission, the ECB and the IMF. The oversight of these international organizations reduces the margin for data inaccuracies or misreporting, but nevertheless the Greek GDP-linked warrants were also heavily discounted (see Figure \ref{Figures_Taehoon}).

The theoretical values for the Argentina GDP-warrant estimated in \citet*{ChamonCostaRicci2008} were significantly above those in market reports. This valuation discrepancy was mainly related to the assumptions regarding the growth process. While their calculations were based on the average of the Consensus Forecast, \citet*{ChamonCostaRicci2008} also state that investment banks’ pricing models exhibited more conservative growth forecasts. This suggests that investors’ forecasts of GDP growth were in general pessimistic and underpredicted Argentina's economic recovery. Finally, \citet{CantorPacker1996} argues that financial markets regard sovereign ratings with skepticism, and \citet{GrossePodstawski2017} provide some more recent evidence of investor pessimism and ambiguity-aversion in sovereign debt markets, in line with \citet{PouzoPresno2016}.

Our framework with robustness captures investors' pessimism when pricing sovereign debt. The discrepancy in growth forecasts reflects that investors have not settled on a `true' model governing the data and/or are concerned about specification errors behind the processes used to model economic variables.\footnote{As mentioned before, our framework could also be reinterpreted to accommodate concerns about data reliability and misreporting. If potential misreporting drives a wedge between reported and actual data, reported data could be conceptualized as being generated by a perturbation of the statistical process governing actual data. External scrutiny and other forces discussed above would limit the size of this perturbation, which in our framework maps into a cap on the contribution of these sources to the overall robustness parameter.} Thus, robust lenders charge an uncertainty premium by slanting probabilities towards worse but plausible scenarios to guard themselves against possible errors in the estimated process.

%provides support for our robust lenders who would charge an uncertainty premium by slanting probabilities towards worse but plausible scenarios. %Moreover, modelling lenders who fear model misspecification also captures the potential concerns about data reliability and misreporting discussed above.



%As was the case with other similar instruments, the Argentinian GDP warrants traded at heavy discounts. Pricing bonds with indexed repayments requires taking expectations and therefore a model for the distribution of the stochastic process upon which the payments are contingent. \citet*{ChamonCostaRicci2008} use Monte Carlo simulations based on historical data to compute theoretical prices for the GDP warrants we are interested in. They find wide spreads: between 300 and 400bps which they attribute to the default risk of other securities, and a residual of 1200bps (which declined over time to about 600bps) which they interpret as a premium for `novelty.' As we will see below, our framework with robust lenders is able to generate deep discounts, in line with the empirical evidence. In this sense, we interpret the wide spreads on some types of sovereign state-contingent debt as reflecting ambiguity premia.


