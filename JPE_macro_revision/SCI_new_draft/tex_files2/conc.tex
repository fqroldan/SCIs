\section{Conclusion}\label{sec:conc}

This paper studies why experiences with sovereign state-contingent debt instruments have not had the success anticipated by a literature which highlights their risk-sharing benefits. We rationalize the scarce popularity of these instruments in the context of a standard sovereign default model à la \cite{Eaton81} with long-term debt in which foreign investors have concerns about model misspecification. International lenders are ambiguity-averse and guard themselves against possible misspecification errors in their approximating model by cautiously approaching (and pricing) the bonds offered by the issuing country. While state-contingent debt is effective in reducing default risk, robust lenders distort probabilities by assigning higher likelihood to those states where the bond promises lower repayments, resulting in an ambiguity premium associated with the contingency of the bond.

We show that this ambiguity premium can be very large when state-contingent bonds feature the threshold structure observed in recent issuances by emerging markets (e.g., Argentina, 2005; Greece, 2012; Ukraine, 2015), which results in substantial welfare losses. This bond structure, embedded in our model with robust international lenders, can account for the little use of these financial instruments and their unfavorable pricing. However, even this `threshold' bond generates welfare gains when facing rational-expectations lenders. In this regard, we also show how the optimal bond design crucially depends on the degree of the lenders' preference for robustness.

The optimal design of state-contingent debt with robust lenders balances several forces. Lenders charge premia for ambiguity related to the stipulated payments (ex-ante contingency) as well as for ambiguity related to default (ex-post contingency). As defaulting is costly, the optimal design uses ex-ante contingency to mitigate or eliminate the probability of default ex-post. When lenders have an extreme degree of robustness, the government designs a bond that eliminates as much contingency as possible. In intermediate cases, the optimal structure enables some probability distortions in order to provide risk-sharing. The results of our calibration exercise generally support a state-contingent structure with linear indexation and potentially a threshold to cover against the extreme left tail of shocks to income.