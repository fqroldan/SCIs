\ifdefined\ungated
\else
	\def\ungated{1}
\fi

\ifnum\submission=1%
\else
  \section{Introduction}
\fi

The recent European sovereign debt crises and the increase in public debt levels after the COVID-19 shock have brought proposals for state-contingent debt instruments to the forefront of policy debates as a strategy to avoid costly defaults (\citealp{UN2006}; \citealp{Blanchard2016}; \citealp{IMFpolicy2017}; \citealp{IMF2020}). There is also a substantial theoretical literature focusing on the merits of indexing sovereign debt to real variables to help with macroeconomic stabilization and risk sharing. \cite{Shillerbook}, \cite{HMindexed2012}, \citet*{voxeubook} and \cite{KimOstry2021} argue that GDP-indexed bonds could allow governments to reduce both the cyclicality of fiscal policy and default risk while improving risk sharing with international creditors. More generally, the benefits of improving international risk sharing have been discussed extensively since the seminal work by \cite{BackusKehoeKydland92}. Several recent studies focusing on the advantages of fiscal unions have found that the gains from improving regional stabilization and risk sharing are quantitatively important (\citealp{Beraja2020}; \citealp{FarhiWerning2017}). \cite{ObstfeldPeri1998} argue that state-contingent debt could replicate these features without having to resort to a politically unfeasible combination of taxes and transfers.\footnote{\cite{Beraja2020} and \cite{FarhiWerning2017} show that the efficient risk sharing arrangement within a fiscal union could be achieved through a simple contingent transfer rule that bear resemblances to a GDP-indexed bond.}

Despite these well-understood advantages, the use of state-contingent debt instruments is scarce in practice and countries have not been able to issue such financial instruments at a reasonable premium---as in the recent cases of Argentina (2005), Greece (2012) and Ukraine (2015). Surprisingly, while some practical implementation challenges have been discussed among policy makers, there is little theoretical analysis investigating them and the lack of indexation in sovereign debt markets remains a puzzle. \citet{IMFpolicy2017} and \citet{voxeubook} point to myopia on the part of issuers, who might be out of office before the gains fully materialize.\footnote{\cite{Amador2012} and \cite{AguiarAmadorFourakis2020} study sovereign debt models including political myopia that could arise because of political polarization or political turnover.} \cite{Krugman88} argues that GDP-indexed bonds could create moral hazard problems by deincentivizing the government to conduct growth-friendly policies or misreport GDP statistics. However, these arguments do not seem to be empirically relevant.\footnote{This argument should also apply to inflation-linked bonds but many countries have issued these type of securities. Moreover, the moral hazard problem could also apply to non-contingent debt if the costs of defaults vary with income as is often assumed in the sovereign default literature.} Others argue that markets for these instruments tend to be shallow and, thus, these bonds would carry a large liquidity premium. \cite{Moretti2020} investigates this liquidity channel and finds that state-contingent debt is still welfare-improving. Overall, there are no compelling arguments in the literature to outweigh the aforementioned merits of indexation and justify their little use in practice.

This paper aims to fill this gap and proposes a novel mechanism to understand why state-contingent debt has only been issued on a modest scale and severely underpriced. We evaluate state-contingent instruments in light of a sovereign default framework à la \cite{Eaton81} with long-term debt, augmented with international lenders who fear model misspecification.\footnote{\cite{JIE} show that allowing for long-term debt introduces a time inconsistency problem (debt dilution) that improves the quantitative performance of sovereign default models. If the government only issues one-period bonds, there is no time inconsistency in borrowing decisions (\citealp{AguiarAmador2019}).} Since the seminal contribution of \citet{Arellano05}, the Eaton-Gersovitz model has been widely used in studies of fiscal policy for countries with default risk. In our environment, foreign lenders have doubts about the probability model of the small open economy's exogenous income process and guard themselves against this ambiguity by forming pessimistic expectations. Lenders with preferences for robustness distort probabilities about exogenous shocks in an endogenous way by boosting the probability of low-utility states and seek decision rules that perform well under these worst-case distributions. In the model, low utility events are associated with periods of high default risk and, in the case of state-contingent debt, periods in which stipulated repayments are low.

The presence of robust lenders creates a tradeoff for the optimal design of state-contingent debt. Obtaining more insurance from lenders increases the relevance of their doubts about the true stochastic process governing the indexing variable. On the other hand, contingency in stipulated repayments can help reduce the probability of default and the volatility of consumption while improving risk sharing with the foreign creditors as in previous studies \citep[e.g.,][]{HMindexed2012}. The typical design of state-contingent instruments that countries have recently used involves a threshold below which no payments are made (Argentina 2005, Greece 2012, and Ukraine 2015 are 21st century examples). This threshold structure, especially when the threshold is imposed at relatively high levels of income, is sensitive to the types of probability distortions that robust lenders worry about, which leads them to heavily discount the bond.\footnote{In contrast, many countries implicitly index their debt portfolio through a combination of local-currency, foreign-currency, and inflation-indexed debt, which builds in state-contingency but avoids thresholds. Low spreads in such cases is consistent with our model, where linear indexation is generally superior to threshold indexation.} Threshold indexation also provides little insurance in exchange for the repayment variance it imposes on lenders. In line with the empirical evidence, our model with robust lenders generates wide spreads for this threshold-type of instruments, which also lead to equilibrium welfare losses compared to noncontingent debt and, thus, explains why governments seem so reluctant to issue these instruments.

Robustness is a standard device in the asset-pricing literature which enables more realistic market prices of risk. In the context of noncontingent debt, \cite{PouzoPresno2016} show that augmenting the baseline sovereign default model with robust lenders is essential to simultaneously match the spread dynamics and the frequency of default observed in the data. They also show that the same model with lenders with standard CRRA preferences and no robustness would generate counterfactually high bond prices for plausible levels of risk aversion. Similarly to the equity premium puzzle, for large values of risk aversion the model without robustness can generate high spreads for noncontingent debt at the expense of an extremely low risk-free rate at odds with the data.

We motivate this framework as a way to capture concerns that the model used to fit the past GDP series, or the underlying variable associated with the state-contingent bond, does not (yet) accurately capture all relevant features of the economy. While we do not pursue them, there are other reasons why the robust-lenders model can be appealing. One is that robustness captures potential concerns that the evolution of the country's GDP may be different in the future than the current estimate of what it has been in the past. It could be that, as a result of (the success of) moving to state-contingent debt, the government modifies its fiscal or monetary policies. It could even be that reported GDP and actual GDP diverge because of incentives to misreport statistics. Finally, it could be that issuing state-contingent debt signals some underlying type of the government: maybe an impatient one who seeks to finance irresponsible policies, or a responsible one who seeks to improve risk-sharing with the rest of the world. While we would prefer explicit models of most of these features, robustness could in principle be interpreted as representing some of them, informally capturing the degree of credibility that lenders assign to the countries they lend to.

The sovereign default framework à la \cite{Eaton81} is commonly used for quantitative studies of sovereign debt and has been shown to generate a plausible behavior of sovereign debt and spreads. Formally, we analyze a small open economy that receives stochastic endowments of a single tradable good. The government is benevolent, issues long-term debt in international markets, and cannot commit to repay its debt. While not in default, the government issues debt which is purchased and priced by competitive foreign lenders. Following \cite{PouzoPresno2016}, we extend the canonical model by assuming that these foreign lenders are endowed with multiplier preferences \citep{HansenSargent2001}, a tractable way to introduce concerns for robustness. In our baseline model, the government issues noncontingent bonds. By varying the asset structure, we examine the equilibrium consequences of making debt payments linked to the realization of the income shock in different ways.

We structure our discussion around the GDP warrants issued by Argentina as part of its 2005 debt restructuring.  \citet*{ChamonCostaRicci2008} find that these bonds traded at large premia: around 300bps which they attribute to the default risk of other securities, and a residual of between 600 and 1200bps which they interpret as a premium for `novelty.' We first calibrate our model to match key moments in the data for Argentina assuming that the government only issues noncontingent debt. Then we evaluate the effects of indexation by assuming the government can issue a state-contingent bond (which we label as `threshold' bond) that resembles the structure of the GDP warrant issued by Argentina. With rational-expectations lenders, this threshold bond provides welfare gains to the country relative to noncontingent debt. The threshold bond also realizes about two-thirds of the welfare gains of optimal state-contingent debt when facing rational-expectations lenders. These welfare gains, however, are overturned with robust lenders, who charge high spreads as their probability distortions magnify the likelihood of states with lower payments---an ambiguity spread. This ambiguity spread explains most of the so-called novelty premium on Argentine GDP warrants and, more broadly, the lack of appetite for issuing these instruments.

We then characterize the structure of the optimal state-contingent bond and show how it is affected by the degree of robustness.  In contrast to the commonly used threshold bond, the optimal design generates substantial welfare gains, although these gains are decreasing in the level of robustness. For tractability, we first characterize the optimal design of state-contingent debt in a stylized version of our model. We show how mean-preserving differences in the structure of promised repayments --which have no impact on rational-expectations, risk-neutral lenders-- may imply large differences in the spreads charged by lenders who fear model misspecification. The lenders' robustness limits the scope for risk-sharing in particular ways: the optimal debt structure features less contingency, lower slopes, and an avoidance of regions with zero or low stipulated repayments. For a given level of insurance, the government would like to minimize the contingency in stipulated repayments in order to prevent costly probability distortions. But the degree of hedging it can attain is itself limited by default risk ex-post.\footnote{\citet*{DvorkinSanchezSaprizaYurdagul} argue that restructurings introduce some state-contingency ex-post.} As the insurance-cost schedule moves out with robustness, welfare gains from optimal state-contingent are decreasing in the degree of robustness. These insights are preserved in the quantitative version of our model, where for computational reasons we only optimize over a parametric family of state-contingent instruments.

Overall, our findings cast doubts about the desirability of using the threshold type of state-contingent bonds that countries have been issuing in the past and demonstrate how the optimal bond indexation depends on the degree of lenders' preferences for robustness. Our model rationalizes the so-called novelty premia in threshold bonds as ambiguity premia associated with the type of contingency these bonds introduce. These premia overturn the well-understood advantages of indexed bonds (which are present in our model) and create substantial welfare losses for the government. Robustness can therefore explain why state-contingent bonds with GDP thresholds for repayment have not been generally regarded as successful by their issuing countries.

\paragraph{Discussion of the Literature} Our analysis builds on and extends three branches of the literature: sovereign default, robust control theory, and the implications of state-contingent debt. First, our study is related to the recent literature on quantitative models of sovereign default that extended the approach developed by \cite{Eaton81}, starting with \cite{AG_06} and \cite{Arellano05}. Different aspects of sovereign debt dynamics and default have been analyzed in these quantitative studies. Excellent surveys of the literature on sustainable public debt and sovereign default can be found in \citet*{AguiarAmador2021} and in handbook chapters by \citet{AguiarAmadorHandbook}, \cite*{ACCS}, and \cite*{DerasmoMendozaZhang}.

Our study also relates to the literature on robust control methods pioneered by \cite{HansenSargent2001,HansenSargent2011}. A growing theoretical macro literature extends canonical models to the case in which the social planner and/or private agents fear model misspecification and search for robust policies under worst-case scenarios (\citealp{AdamWoodford2012}, \citealp{Young12}, \citealp{FerriereKarantounias},  \citealp{MontamatRoch}). We relate closely to \cite{PouzoPresno2016} who study a sovereign default model with robust international lenders in the context of noncontingent debt. Their analysis demonstrates how the introduction of robust lenders improves the quantitative performance of sovereign default models. Robustness helps match bond spreads dynamics observed in the data without resorting to counterfactually high default frequency by historical standards. We then study how international investors' concerns about model misspecification affect the spreads, welfare implications, and optimal design of state-contingent bonds. 

Finally, our paper naturally relates to a literature concerned with the implications of state-contingent debt. \cite{BorenszteinMauro2004} focus on the implications and benefits of state-contingent debt for the cyclicality of fiscal policy. \cite{Durdu} shows that the degree of indexation should be optimally chosen to smooth sudden stops, and that this optimal degree of indexation depends on the persistence and volatility of the shocks an economy faces. More closerly related to our paper, \cite{HMindexed2012} and \cite{SandlerisSaprizaTaddei2017} study the effects of introducing income-indexed bonds into standard quantitative sovereign default models. Both studies find that, in models without robustness, GDP-indexed securities support large welfare gains when designed optimally. These papers emphasize that GDP-linked bonds allow the government to eliminate default risk while increasing indebtedness, thereby reducing the equilibrium volatility of consumption relative to income. However, their baseline model with one-period bonds generates counterfactual bond spread dynamics and debt levels. For example, the benchmark calibration in \cite{HMindexed2012} generates a 3\% mean spread and 4\% debt-to-income ratio in their simulations. We build on these papers by clarifying how robustness on the part of lenders helps match empirical (low) bond prices which ultimately overturns the conclusions on welfare implications of state-contingent debt for the designs observed in reality.

\paragraph{Layout} The remainder of the paper is structured as follows. First, section \ref{sec:summarySCI} documents recent country experiences with sovereign state-contingent bonds and motivates the mechanism proposed in the paper. Section \ref{sec:2permodel} lays out a simple two-period model that illustrates how the optimal design of state-contingent bonds and the associated welfare implications depend on the lenders' preferences for robustness. Section \ref{sec:quantmodel} introduces the quantitative model. Section \ref{sec:quantresults} contains our main results on the equilibrium effects of different state-contingent debt instruments. Finally, section \ref{sec:conc} concludes.
 
