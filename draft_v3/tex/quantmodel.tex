\section{The quantitative model \label{sec:quantmodel}}

In this section we present the infinite-horizon version of the two-period model studied in Section \ref{sec:2permodel}. We consider a general formulation in which the government can issue contingent defaultable debt and discuss how this nests the benchmark case with noncontingent debt as a special case. %We present the general model in which the government can issue contingent defaultable debt, and we discuss what parametrization would imply the benchmark case with noncontingent debt.

\paragraph{Endowment} There is a single tradable good. The economy receives a stochastic endowment stream of this good $x_{t}$, which is dictated by persistent and transitory shocks. The transitory component of the endowment process $m_t$ is an \emph{iid} shock.\footnote{This random variable is introduced for computational issues when solving the model as discussed in \cite{Chatty}. Because it is only part of the model for numerical reasons, we assume for simplicity that lenders do not have doubts about the distribution of $m_t$.} The persistent component of the endowment process $y_t$ follows a Markov process and takes values in the set $\mathcal{Y} = \left\{ y_1, ..., y_J \right\}$. While the true density of this process is unknown, the government trusts that the evolution of $y_t$ is governed by the approximating model with probabilities $\mathbb{P}(y_{t+1} = y_j \mid y_{t} = y_i)> 0$ for all $i,j=1,...,J$.  

\paragraph{Government} The government's objective is to maximize the present expected discounted value of future utility flows of the representative household in the economy, namely
\[\mathbb{E}_t \left[ \sum_{s=0}^{\infty} \beta^{s} u\left( c_{t+s} \right)\right],\]
where $\mathbb{E}$ denotes the expectation operator, $\beta$ denotes the subjective discount factor, and $c_t$ represents household's consumption. The utility function is strictly increasing and concave.


As in \cite{JIE} and \cite{ArellanoRamanarayanan}, we assume that a bond issued in period $t$ promises an infinite stream of coupons, which decreases at a constant rate $\delta$. However, we expand this framework by making coupon payments vary with the realization of $y_t$. We allow coupon payments to depend linearly on $y_t$ as well as threshold-based rules. In particular, a bond issued in period $t$ promises to pay $\max \{0, (1+\alpha(y_t-1)) \ind (y_t > \tau) \}$ units of the good in period $t+1$ and $\max \{0, (1-\delta)^{s-1} (1+\alpha(y_s-1)) \ind (y_s > \tau) \}$ units in period $t+s$, with $s \geq 2$. The parameters ${\alpha, \tau}$ determine the degree of linear indexation and coupon repayment income threshold, respectively. This bond structure allows us to keep the same recursive formulation of the model presented in the literature. Note that when $\alpha=0$ and $\tau=-\infty$ we recover the benchmark model with noncontingent bonds.\footnote{As in Section \ref{sec:2permodel}, we abstract from the portfolio problem and assume that only one debt instrument is traded. In line with previous studies \citep{BorenszteinMauro2004, Durdu, HMindexed2012, HMSP16, SandlerisSaprizaTaddei2017, KimOstry2021, SosaPadillaStruzenegger2021}, we compare the cases where the government issues only either noncontingent debt or state-contingent debt.}

At the beginning of each period in which it is not in default, the government makes two decisions. First, it decides whether to default. Second, if it chose to repay, it chooses the number of bonds that it purchases or issues in the current period. We follow the literature and assume that, because of acceleration and cross-default clauses, the government cannot discriminate among its creditors or otherwise engineer a partial default. We also assume that the fraction of the loan lenders can recover after a default (the recovery rate) is zero.

There are two costs of defaulting. First, a defaulting sovereign is excluded from capital markets. In each period after the default
period, the country regains access to capital markets with probability $\psi \in \left[ 0,1\right] $.\footnote{\cite*{EQ_august} solve a baseline
model of sovereign default with and without the exclusion cost and show that eliminating this cost affects significantly only the debt level generated by the model.}  Second, if a country has defaulted on its debt, it faces an income loss of $\phi \left(y\right)$ in every period in which it is excluded from capital markets.  

\paragraph{Lenders} Our departure from the standard setup is to allow for lenders who fear model misspecification. We follow \cite{PouzoPresno2016} by considering that the lenders’ have per period payoff linear in consumption, while also being ambiguity averse with respect to the probability distribution of $y_t$.\footnote{As in the stylized model, this assumption makes the size of the lenders' endowment (relative to the small open economy) irrelevant.} Unlike the government, lenders distrust the approximating model. They look for decision rules that are robust to possible errors in the estimated process, by surrounding the approximating density with other densities offering a similar fit to the data and choosing a sequence of distorted conditional probabilities that minimize their expected utility. In particular, we adopt the \cite{HansenSargent2001} multiplier preferences model, which captures ambiguity aversion by a single parameter, to derive a theoretically-founded measure of lenders' ambiguity aversion. In this framework, model uncertainty generates a risk premium without the need for correlation between foreign investors' wealth and default. Foreign lenders charge a premium on defaultable debt in order to guard themselves against possible specification errors in the estimated income process. 


\subsection{Recursive formulation\label{Recursive}}

Let $b$ denote the number of outstanding coupon claims at the beginning of the current period, and $b'$ denote the number of
outstanding coupon claims at the beginning of the next period. A negative value of $b$ implies that the government was a net issuer
of bonds in the past. Let $d$ denote the current-period default decision. We assume that $d$ is equal to $1$ if the government
defaulted in the current period and is equal to $0$ if it did not. Let $V$ denote the government's value function at the beginning of a
period, before the default decision is made.  Let $V_0$ denote the value function of a sovereign not in default.  Let $V_1$
denote the value function of a sovereign in default.  Let $F$ denote the conditional cumulative distribution function of the next-period
endowment $y'$. For any bond price function $q$, the function $V$ satisfies the following functional equation:

\begin{equation}
V \left(b,y,m\right) =  \max_{d\epsilon\{0,1\}} \{d V_1(y,m) + (1-d)
V_0(b,y,m) \},\label{V}
\end{equation}
where
\begin{equation}
V_1(y,m)  =  u\left( y + m -  \phi \left( y\right) \right) + \beta \int
\left[ \psi V(0,y^{\prime },m^{\prime }) +\left( 1-\psi \right) V_1(y',m') \right]
F\left(dy^{\prime }\mid y\right), \label{V1}
\end{equation}

\begin{equation} \label{V0}
\begin{aligned}
&V_0(b,y,m)  =  \max_{b^{\prime } \leq 0} \left\{u\left(c \right) + \beta \int V(b^{\prime },y^{\prime },m^{\prime })F\left( dy^{\prime }\mid y\right) \right\} \\
&\text{subject to }\;
 c = y + m + b\max \{0,  (1+\alpha(y-1)) \ind (y > \tau) \} - q(b',y) \left[ b^{\prime } -  (1-\delta) b \right].
 \end{aligned}
\end{equation}

Let $W$ denote the lenders' value function at the beginning of a period, before the default decision is made. The problem of a robust lender that fears model misspecification can be expressed in recursive form as
\begin{equation} \label{eq:W}
      \begin{aligned}
	&W(b,y) = \max c^L - \frac{\beta^L}{\theta} \log \left( \mathbb{E}\left[ \exp(-\theta W(b',y')) \right] \right) \\
	&\text{subject to } \\
	&c^L = \bar{z} + (1-d(b,y,m)) [q(b',y)(b'-(1-\delta) b) - b\max \{0,  (1+\alpha(y-1)) \ind (y > \tau) \}],
	\end{aligned}
\end{equation}
where the parameter $\theta$ encapsulates the degree of lenders' ambiguity aversion.

In this framework, bond prices are such that uncertainty-averse lenders make zero profits in expectation given their subjective beliefs. The bond price is given by the following functional equation:
\begin{eqnarray}
q(b^{\prime },y) &=&\int M(b^\prime, y^{\prime },y)\left[ (1-h\left(b^{\prime}, y^{\prime}, m' \right))\max \{0,  (1+\alpha(y-1)) \ind (y > \tau_y) \} \right] F\left( dy^{\prime }\mid
y\right) \notag
\\
&&+\left(1-\delta\right) \int M(b^\prime, y^{\prime },y)\left[ 1-h\left(
b^{\prime}, y^{\prime}, m' \right) \right] \;q(g(b^{\prime },y^{\prime }, m'),y^{\prime })F\left( dy^{\prime }\mid
y\right), \label{q}
\end{eqnarray}%
where $h$ and $g$ denote the future default and borrowing rules that lenders expect the government to follow.  The default rule $h$ is equal to $1$ if the government defaults, and is equal to $0$ otherwise. The function $g$ determines the number of coupons that will mature next period. The first term in the right-hand side of equation (\ref{q}) equals the expected value of the next-period coupon payment promised in a bond. The second term in the right-hand side of equation (\ref{q}) equals the expected value of all other future coupon payments, which is summarized by the expected price at which the bond could be sold next period. The lenders' stochastic discount factor, $M(b^\prime, y^\prime,y)$, is made of two parts. First, an ordinary discount factor $\beta^L$ that applies in cases without model uncertainty. Second, an ambiguity-aversion factor, which is given by the conditional likelihood ratio of the endogenous worst-case distorted model relative to the approximating model
\begin{equation}
M(b^\prime, y^{\prime },y) = \beta^L \frac{\exp\left(-\frac{W(b',y')}{\theta}\right)}{\mathbb{E}\left[\exp\left(-\frac{W(b',y')}{\theta}\right)\right]}
\end{equation}

Equations (\ref{V})-(\ref{q}) illustrate that the government finds its optimal current default and borrowing decisions taking as given
its future default and borrowing decision rules $h$ and $g$. In equilibrium, the optimal default and borrowing rules that solve
problems (\ref{V}) and (\ref{V0}) must be equal to $h$ and $g$ for all possible values of the state variables.

\begin{definition}
A Markov Perfect Equilibrium is characterized by
\end{definition}

\begin{enumerate}
	\item a set of value functions $W$, $V$, $V_1$, and $V_0$
	
	\item a default rule $h$ and a borrowing rule $g$,
	
	\item a bond price function $q$,
\end{enumerate}

such that:

\begin{enumerate}[(a)]
	\item given $h$ and $g$, $V$, $V_1$, $V_0$, and $W$
	satisfy functional equations  (\ref{V}),  (\ref{V1}), (\ref{V0}), and (\ref{eq:W}) when the government can trade bonds at $q$; 
	\item given $h$ and $g$, the bond price
	function $q$ is given by equation (\ref{q}); and
	\item the default rule $h$ and borrowing rule $g$ solve the
	dynamic programming problem defined by equations (\ref{V}) and
	(\ref{V0}) when the government can trade bonds at $q$.\footnote{\cite{AguiarAmador2020} show that in an Eaton-Gersovitz model with long-term debt, there may be multiple MPE equilibria. We rule out this possibility by focusing on the equilibrium that is the limit of the equilibrium of the finite-horizon economy.}
\end{enumerate}


\subsection{Calibration\label{calibration}}

As in many previous quantitative studies on sovereign default (e.g., \citealp{Arellano05}, \citealp{JIE}, \citealp{Chatty}, \citealp{PouzoPresno2016}), we use Argentina before the 2001 default as a case study. In particular, we focus on the time period starting when Argentina regained access to international capital markets (1993:I) and ending when it defaulted on its external debt (2001:IV). We solve the model numerically using value function iteration and using a discrete state space technique. Following \cite{Chatty}, we apply randomization methods to overcome convergence problems that arise in models of sovereign default with long-term debt. See Appendix \ref{app:algorithm} for a detailed description of the numerical algorithm.

The utility function is assumed to display a constant coefficient of relative risk aversion denoted by $\gamma$. That is, \begin{equation*} 
u\left( c\right) = \frac{c^{ 1-\gamma }-1}{1-\gamma }.
\end{equation*}

The persistent component of the endowment $y_{t}$ follows an AR1 process in logs
\begin{equation*}
\log (y_{t})=\rho \log(y_{t-1})+\varepsilon _{t},
\end{equation*}
with $|\rho |<1$, and $\varepsilon _{t}\sim \mathcal{N}\left( 0,\sigma_{\epsilon }^{2}\right)$. The distribution of the transitory component of the endowment is truncated between $\left[-2 \sigma_{m},2 \sigma_{m}\right]$, with $m_{t}\sim \mathcal{N}\left( 0,\sigma_{m}^{2}\right)$. Following \cite{Chatty}, we assume a quadratic loss function for income during a default episode $\phi \left( y \right) = max \{ 0, d_0 y + d_1 y^2 \}$, which they show it helps the model's ability to account for the dynamics of the sovereign debt interest rate spread.

\newcolumntype{G}{>{\bfseries\centering\arraybackslash}m{0.8in}}
\begin{table}[!hbtp]\centering \small
\caption{Parameter values for the baseline parametrizations.}\label{Table_parameters}
\begin{tabular}{@{}lccc@{}} \toprule
 & \textbf{Parameter} &  \multicolumn{1}{G}{\textbf{Benchmark Values}} \\\midrule
Sovereign's risk aversion   & $\gamma$  & 2 \\
Interest rate & $r$   & 0.01 \\
Income autocorrelation coefficient & $\rho $  & 0.9484 \\
Standard deviation of $y_{t}$ & $\sigma_{\epsilon }$  & 0.02\\
Standard deviation of $m_{t}$ & $\sigma_{m}$  & 0.03\\
Reentry probability & $\psi $  & 0.0385 \\
Duration of debt & $\delta $  & 0.05 \\
Discount factor & $\beta$  & 0.9627 \\
Default cost: linear  & $d_0$   & -0.255 \\
Default cost: quadratic  & $d_1$    & 0.296 \\
Degree of robustness & $\theta$ & 1.62   \\
Linear coupon indexation & $\alpha$ & 0 \\
Coupon repayment threshold & $\tau$  & $-\infty$  \\
\bottomrule
\end{tabular}
\end{table}

Table \ref{Table_parameters} presents our benchmark calibration.\footnote{We follow the calibration strategy of \cite{PouzoPresno2016}.} A period in the model refers to a quarter. During the time period under consideration, Argentina only issued noncontingent debt, which in our framework implies setting $\alpha=0$ and $\tau  =-\infty$. The representative household in the sovereign economy has a coefficient of relative risk aversion of 2, which is standard in studies of business cycles. The risk-free interest rate is set to 1 percent. Parameter values that govern the endowment process are chosen so as to mimic the behavior of GDP in Argentina from the first quarter of 1993 to the last quarter of 2001. The parametrization of the income process is similar to the one used in other studies that consider a longer sample period (see, for instance, \citealp{AG_06}). We assume a probability of regaining access to capital markets that implies an average period of 6.5 years of financial exclusion, consistent with the estimates in \cite{Benjamin}.  Following \cite{Chatty}, we target an average maturity of $5$ years for noncontingent bonds by setting $\delta = 0.05$.

There are four remaining parameters: the discount factor ($\beta$), the two parameters that define the output cost of defaulting ($d_0, d_1$), and the degree of robustness ($\theta$). These parameter values are calibrated to match four moments in the data: (i) the level of external government debt to quarterly output ratio (46 percent); (ii) the average interest rate spread (8.15 percent); (iii) the standard deviation of the spread (4.58 percent); and, (iv) a default frequency of 3 defaults per 100 years. Argentina defaulted on its external debt five times since 1824 (see \cite{Yue05}), which implies an annual default frequency of 2.5\%. Even though it is not clear which data values for the default frequency one should target, we choose to target this statistic because it received considerable attention in the literature. For example, \cite{AG_06}, \cite{Arellano05}, \cite{nota}, \cite{Lizarazo_default} and \cite{PouzoPresno2016} all target an annual default frequency of 3 percent. The targets for the spread distribution are taken from the spread behavior in Argentina before its 2001 default. The target for the mean debt to GDP ratio consists of the average public external debt between 1993 and 2001.





