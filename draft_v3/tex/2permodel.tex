\section{A stylized model of sovereign default with robustness \label{sec:2permodel}}

This section presents a stylized sovereign default model to conceptually explore the interaction between state-contingent debt and robustness. A small open economy, populated by a government and a representative agent, faces risk-neutral competitive foreign lenders. The world lasts for two periods in which the government receives endowments $(y_1, y_2)>0$. There is uncertainty about $z$ which determines the value of income in the second period $y_2(z)$.

\paragraph{Assets} Only one type of security is traded.\footnote{In line with previous studies \citep{BorenszteinMauro2004, Durdu, HMindexed2012, HMSP16, SandlerisSaprizaTaddei2017, KimOstry2021, SosaPadillaStruzenegger2021}, we abstract from the portfolio problem and compare the cases of noncontingent and contingent debt. While in reality different types of bonds might be issued simultaneously, the set of all state-contingent bonds is large and contains objects such as a share $x$ of threshold bonds with threshold $\tau$ and $1-x$ of noncontingent bonds. In this two-period example, however, the optimal bond does not resemble such a mixture.}
When the government issues debt, it promises a repayment $R(z)$ in state $z$ of the second period. Different specifications of the stipulated repayment function $R$ represent different types of state-contingent debt structures. We focus on four different types of repayment functions summarized in Table \ref{tab:R(z)}.

\begin{table}[!hbtp]\centering \small
\caption{Stipulated repayment functions}\label{tab:R(z)}
\begin{tabular}{@{}lccc@{}}\toprule
  \textbf{Type of debt} & \multicolumn{3}{c}{\textbf{Stipulated repayment}} \\\midrule
  Noncontingent debt & $R(z)$ & $ = $ & $1$ \\
  Linear indexing & $R^\alpha(z)$ & $ = $ & $1 + \alpha \left( y_2(z) - \mathbb{E}_1\left[y_2(z)\right] \right) $ \\
  Threshold debt & $R^\tau(z)$ & $ = $ & $\ind\left(y_2(z) > \tau\right)$ \\
  Optimal design & $R^\star(z; \theta)$&       & {\footnotesize chosen state by state}\\\bottomrule
\end{tabular}
\end{table}
Noncontingent debt promises a constant repayment regardless of the state, while the repayment of linearly-indexed debt depends on the difference between realized output and its mean, with a slope parameter of $\alpha \geq 0$.\footnote{In the model, we use $\max\{0,R^\alpha(z)\}$ when considering linearly-indexed debt to keep repayments nonnegative. We skip the $\max$ in the description to keep the notation succint.} Threshold debt pays only if the state is above a minimum level. Finally, we compute the debt structure that maximizes the utility of the government by promising non-negative repayments $R^\star$ state-by-state in a flexible manner. Our notation anticipates that the optimal design depends on the lenders' preferences as summarized by the robustness parameter $\theta$ introduced below.

\paragraph{Government} The government is benevolent and makes decisions on a sequential basis. The government acting in period $j \in \{1,2\}$ maximizes $\mathbb{E} \left[ \sum_{t=j}^2 \beta_b^{t-j} u\left( c_t \right)\right]$, where $\mathbb{E}$ denotes the expectation operator, $\beta_b \in (0,1]$ is the government's discount factor, $c_{t}$ represents period-$t$ consumption in the economy, and the utility function $u$ is increasing and concave. The government borrows to finance consumption in period 1, taking as given the stipulated repayment function $R$. The government may choose to default in period 2, in which case it does not pay the debt but loses $\phi(z)$ of the endowment $y_2$. We consider a standard quadratic specification for the output-cost function, meant to make the cost of default increasing and convex in output
\begin{align*}
  \phi(z) =  d_1 y_2(z)^2
\end{align*}

% \paragraph{Assets} Only one type of security is traded: when the government issues debt, it promises a repayment of $R(z)$ in state $z$ of the second period. We study three versions of the model depending on our assumption for the repayment function $R$. First, we assume the government issues non-contingent bonds. In this case a bond issued in period 1 promises to pay $R(z) = 1$ in period 2 for any realization of $z$. Second, we assume the government can issue state-contingent debt by making the period 2 coupon payment, $R^\alpha(z)$, depend linearly on the realization of $y_2(z)$:
% \begin{align*}
% 		R^\alpha(z) = 1 + \alpha (y_2(z) - 1)
% \end{align*}
% where $\alpha \in [0,1]$ governs the degree of indexation (notice that $\alpha = 0$ recovers an uncontingent bond). Third, we study the optimal indexation where the next period coupon payment $R(z)$ is chosen state by state to maximize the ex-ante value of the government in equilibrium (this case is equivalent to one where the government has access to an Arrow security for each value of $z$ and will in general be different from the case of $\alpha = 1$).

The government understands the pricing function $q(b)$ that foreign lenders offer for an issuance level $b$. For ease of notation, we omit the dependence of $q$ on $R$ and $\theta$. The government's problem is to choose debt and consumption to solve
\begin{align*}
	V(\theta, R) = &\max_b u(c_1^b) + \beta_b \mathbb{E}\left[u(c_2^b) \right] \\
	\text{subject to }\;
	& c_1^b = y_1 + q(b) b \\
	& c_2^b = y_2(z) - \phi(z)d(b,z) - (1-d(b,z)) R(z) b
\end{align*}
where $d(b,z)$ takes the value of 1 if the goverment defaults in state $(b,z)$ and 0 otherwise. $V(\theta, R)$ denotes the equilibrium value attained by the government when it faces lenders with robustness $\theta$ and issues debt with stipulated repayment $R$. It is common knowledge that default takes place if and only if
\begin{align*}
	u\left(y_2(z) - \phi(z)\right) > u\left(y_2(z) - R(z) b\right)
\end{align*}

\paragraph{Lenders} We focus on the interaction between the design of the debt instrument and the lenders' degree of robustness. Following \citet{HansenSargent2001} and \citet{PouzoPresno2016}, we assume that foreign lenders feature \emph{multiplier preferences} to capture concerns about potential model misspecification. Multiplier preferences lead our lenders to price assets by distorting their approximating or benchmark model. 
% \paragraph{Lenders} We also consider different alternatives with respect to lenders. First, we assume that bonds are priced by competitive risk-neutral investors who evaluate future payments with a discount factor $\beta$. Second, following \cite{PouzoPresno2016}, we assume that lenders feature multiplier preferences to capture concerns about model misspecification.
% Ambiguity-averse preferences lead foreign lenders to use distorted probabilities to price assets.
They seek rules that perform well under a variety of possible models that are statistically close to their benchmark. A common metaphor is that lenders choose their actions to maximize utility while a fictitious `evil agent' chooses a probability distribution to minimize that same utility. 
% The minimizing agent chooses a worst-case probability distribution that minimizes the utility resulting from the maximizing agent's choice.
As a result, the action chosen performs well across a range of possible specification errors of the approximating model. The overall utility includes a gain from the entropy of the distribution used with respect to the benchmark model (resulting in an entropy cost incurred by the evil agent), which limits the size of the distortions that are considered. The key parameter is the reciprocal of the marginal cost of relative entropy, $\theta$.\footnote{\citet{PouzoPresno2016} provide a thorough discussion of robustness in the context of sovereign debt models.}

Standard arguments from the robustness literature allow us to write the lenders' problem as\footnote{See Appendix \ref{app:lenders}.}
\begin{align*}
	&\max u(c_1^L) - \frac{\beta}{\theta} \log \left( \mathbb{E}\left[ \exp(-\theta v_2^L) \right] \right) \\
	\text{subject to }\;
	& v_2^L = u(c_2^L) \\
	& c_2^L = w_2 + (1-d(b,z)) R(z) b \\
	& c_1^L = w_1 - q(b) b
\end{align*}
where $(w_1, w_2)$ are the lenders' endowments in periods 1 and 2, respectively.\footnote{In the case of risk-averse lenders, the relative size of their endowments can also be important in shaping their risk-appetite. Moreover, in general, lenders can be affected by developments in the economy through a correlation between these quantities and the endowment shocks.}

The lenders' first-order conditions yield a pricing equation for the debt
\begin{align*}
	u'(c_1^L) q(b) = \beta \mathbb{E}\left[ \frac{\exp(-\theta u(c_2^L))}{\mathbb{E}\left[\exp(-\theta u(c_2^L))\right]} u'(c_2^L) (1-d(b,z)) R(z) \right]
\end{align*}
where $M = \beta \frac{\exp(-\theta u(c_2^L))}{\mathbb{E}\left[\exp(-\theta u(c_2^L))\right]}$ augments the stochastic discount factor. The parameter $\theta$ controls the degree of ambiguity aversion. This Euler equation makes it clear that the model converges back to expected utility with rational expectations as $\theta \to 0$. In our baseline, lenders have per-period payoff linear in consumption, while also being uncertainty averse or ambiguity averse.\footnote{We leave the lenders' utility function general, even though we focus on the risk-neutral case. In general, it can be jointly calibrated along with the robustness parameter $\theta$ to match asset-pricing evidence. Another alternative is to calibrate $\theta$ to a reasonable model error-detection probability.}

Because lenders are risk-neutral, the relative size of their endowment $(w_1, w_2)$ is irrelevant for outcomes \citep[see][for a proof in the context of noncontingent debt]{PouzoPresno2016}. We furthermore set $w_2$ to a constant, attributing all variation in lenders consumption to the small open economy's decisions. However, the small open economy's debt is in reality a small part of their lenders' portfolio. We could capture this fact by making $w_2$ a random variable representing all the payments that lenders receive from their other investments. But as long as $w_2$ is independent of the small open economy's debt repayment and default decisions (which would be the case under diversification), nothing in our equilibrium changes. It is therefore without loss of generality to simplify $w_2$ to be a constant.\footnote{Moreover, in the presence of global shocks affecting the entirety of lenders' portfolio, we would see a positive correlation between the endowments of both parties, which could lead to even more probability distortions than we are considering.}

In other words, the expected actions of the small open economy affect the stochastic discount factor applied to assets whose returns covary with the small open economy's state. In this sense, the small open economy affects the global stochastic discount factor. However, because the small open economy's state is independent of $w_2$, it does not affect the stochastic discount factor applied to assets whose returns do not covary with the small open economy's state. In this sense, the small open economy does not affect the global stochastic discount factor.

\paragraph{Ambiguity premia} The robust-lenders model links bond prices and spreads to features of equilibrium expectations about debt repayment. For risk-neutral (but still robust) lenders, we have
\begin{align}\label{eq:Euler_decomp}
  q(b; R, \theta) &= \beta \mathbb{E}\left[ \frac{\exp(-\theta c_2^L)}{\mathbb{E}\left[\exp(-\theta c_2^L)\right]} (1-d(b,z)) R(z) \right] \notag \\
  &= \underbrace{\beta \mathbb{E}\left[(1-d)R\right]}_{=\, q_{\text{\emph{RE}}}} + \underbrace{\left(1-\mathbb{P}(d)\right)\text{cov}{(M,R)}}_{=\, q_\theta^\text{\emph{cont}}} - \underbrace{\mathbb{E}\left[R\right]\text{cov}{(M,d)}}_{=\, q_\theta^\text{\emph{def}}}
\end{align}

Equation (\ref{eq:Euler_decomp}) breaks up the debt price into a rational-expectations component $q_\text{\emph{RE}}$ and two components that depend on the degree of robustness. The first of them, $q_\theta^\text{\emph{cont}}$, reflects ambiguity in the contingency of the debt contract itself: given the repayment probability, it is proportional to the covariance between the stochastic discount factor $M$ and the stipulated repayment $R$. The second one, $q_\theta^\text{\emph{def}}$, reflects ambiguity in the default strategy: controlling for the average level of stipulated payments, it is proportional to the covariance between the stochastic discount factor and the repayment strategy. Because the lenders' marginal utility is decreasing in debt payments, these covariances will be respectively negative and positive. Both ambiguity terms contribute to lower bond prices and larger spreads.

We compute and decompose spreads as follows. Let $r = \frac{\mathbb{E}[R]}{q}$ be the implicit interest rate and $r-r^\star$ be the spread, where $r^\star = 1/\beta - 1$ is the international risk-free rate. We define the rational-expectations spread as $\text{\emph{spr}}_\text{RE} = \frac{\mathbb{E}[R]}{q_\text{RE}} - r^\star$, the premium from the ambiguity of contingent repayment as $\text{\emph{spr}}_\theta^\text{\emph{cont}} = \frac{\mathbb{E}[R]}{q_\text{RE} + q_\theta^\text{\emph{cont}}} - \frac{\mathbb{E}[R]}{q_\text{RE}}$, and the premium from the ambiguity of default as $\text{\emph{spr}}_\theta^\text{\emph{def}} = \frac{\ex{R}}{q_\text{RE}+q_\theta^\text{\emph{cont}}+q_\theta^\text{\emph{def}}} - \frac{\ex{R}}{q_\text{RE}+q_\theta^\text{\emph{cont}}}$.

Before analyzing the equilibrium probability distortions, it is useful to make the following observation.
\begin{lemma}
  Equilibrium bond prices are (weakly) lower with threshold debt than with noncontingent debt
  \begin{proof}
    Supose that in an equilibrium with uncontingent debt the government issues $b_n$ and therefore defaults in the second period whenever $z < z^d$. This means that noncontingent debt trades at a price of 
    \begin{align*}
      q_d = \ex{ M \ind(z < z^d) }
    \end{align*}

    Now fix $\tau$ and consider the threshold bond that repays when $z \geq \tau$. If $\tau \geq z^d$, the set of states in which the original bond defaulted is contained within the set of states in which the threshold bond pays zero. Consequently, the price of the threshold bond is strictly lower.
    
    If instead $\tau < z^d$ and the government issues the same amount $b_n$ that it did when debt was noncontingent, equilibrium payments are exactly the same. The only difference is that for $z \in (-\infty, \tau)$ the lack of payment would be stipulated in the contract and for $z \in [\tau, z^d)$ there would an ex-post default.
    
    If it issues less, say $b < b_n$, the default set shrinks, but it does so in the same way as when debt was stipulated to be noncontingent (until we hit $\tau$, at which point it stops shrinking). This means that all possibles choices of $b$ were possible at the same price (or better) when debt was noncontingent. Therefore, choosing $b < b_n$ in such a case is ruled out as it violates revealed preference.
  \end{proof}
\end{lemma}

\paragraph{Observation} The measurement and definition of spreads is uncontroversial when debt is noncontingent and/or agents agree on the distribution of shocks. Here both conditions fail, which forces us to decide on a concept of spreads. In our decomposition, all expectations are taken with respect to the government's model (which coincides with the lenders' approximating model and with Nature's model). The implicit interest rate $r$ also takes into account the stipulated contingency $\mathbb{E}[R]$, evaluated in a risk-neutral way under the government's model. A contingent bond which promises to repay in a smaller set of states and hence naturally trades at a lower price only pays higher spreads in our measure if the discount is larger than explained by the `objectively' lower likelihood of repayment.

The robust-lenders model allows us to characterize the probability distortions that underpin debt prices in an equilibrium. Let the \emph{distorted expectation} of a random variable $X$ be the objective expectation of the product of $X$ with a likelihood ratio
\begin{align}\label{eq:kernel}
  \tilde{\mathbb{E}}\left[X\right] = \mathbb{E}\left[\frac{\exp(-\theta u(c_2^L))}{\mathbb{E}\left[\exp(-\theta u(c_2^L))\right]} X \right]
\end{align}
As compared to the expectation taken with the objective probability measure, the distorted expectation magnifies the likelihood of states for which the lenders' utility is low. Different designs for government debt (different $R$ functions) lead to different equilibrium outcomes for lenders, which in turn support different worst-case models and different probability distortions.

\subsection{Probability Distortions}
To investigate the effect of robustness on state-contingent debt prices, we solve the stylized model for different repayment functions $R$ and different levels of the robustness parameter $\theta$. We leverage Equation (\ref{eq:kernel}) to recover the probability distortions used by lenders to evaluate debt payoffs in each equilibrium.

Table \ref{Table_2per_parameters} summarizes our parametrization. We keep close to the Argentinian GDP-linked bonds described in \citet*{ChamonCostaRicci2008}. The parametrization is purposely artificial to highlight how the lenders' robustness interacts with the design of bonds to create spreads. 
In particular, we allow the robustness parameter $\theta$ to vary between $0$ (rational expectations) and $4$, an arbitrary level that is high enough to illustrate the forces at play. Section \ref{sec:quantmodel} focuses on an infinite-horizon version of the model calibrated to Argentina in which $\theta$ is controlled by model detection error probabilities.
\begin{table}[!hbtp]\centering \small
  \caption{Parametrization of stylized model}\label{Table_2per_parameters}
  \begin{tabular}{@{}lcc@{}} \toprule
    \textbf{Parameter} & \textbf{Target} & \textbf{Value} \\ \midrule
    $\beta_b$       & Borrower's discount rate & 6\% ann. \\
    $\beta$         & Risk-free rate & 3\% ann.        \\
    $\gamma$        & Borrower's risk aversion & 2     \\
    $d_1$        & Output cost of default & 20\%    \\
    $g$             & Expected growth rate& 8\% ann.   \\ 
    $\tau$          & Threshold for repayment & 1   \\
    $\sigma_z$      & Std.~deviation of log output & 0.15 \\
    \bottomrule
  \end{tabular}
\end{table}

One period is five years. We set the first period endowment $y_1$ to make $\mathbb{E}[y_2(z)] = 1 = y_1 (1+g)^5$, so that $g$ is the expected growth rate.\footnote{While foreign lenders agree with the second equality, their worst-case model will in general yield a distorted expectation $\tilde{\mathbb{E}}[y_2(z)] < 1$.} The output cost of default $d_1$ as well as expected growth $g$ are set to high values to simultaneously generate high levels of debt and a low default probability, which is complicated in this stylized model by the use of one-period bonds (this difficulty is absent in our quantitative version with long-term debt). Output in the second period $y_2(z) = \exp(\sigma_z z)$ where $z \sim \mathcal{N}(0,1)$. Other parameters are set to standard values in the literature.

We parametrize our threshold bond structure as follows. We set the repayment threshold $\tau$ at the mean of second-period output. This is meant to replicate the fact that the Argentinian GDP-linked bond was designed to pay if output growth was above average. At the time of issuance, the Consensus Forecast for Argentina's GDP growth was about 3\% over the medium-term, which coincides with the bond's main condition for repayment.

\paragraph{Simple state-contingent instruments}
We begin by analyzing equilibrium outcomes associated with some simple bond structures: noncontingent debt, linearly-indexed debt, and threshold debt.

Figure \ref{Figure_distorted_noncontingent} shows the probability distortions when the government issues noncontingent bonds. For ease of exposition, we fix the amount of debt issued at the optimal level when $\theta = 0$ (the rational-expectations case), so that the default probability does not vary with the degree of robustness.\footnote{When the government can optimize issuances as a function of $\theta$, as we will see later, it issues less debt as lenders become more robust and charge higher spreads. This moves the default threshold to the left as $\theta$ increases.} This allows us to concentrate on the effect of robustness on prices, without the compensating reaction of the government. The top panel shows the default probability at each state as a dotted line and the distorted density (used by lenders to evaluate payoffs) in solid lines. The bottom panel shows the stipulated payment $R$ as a dashed line and the likelihood ratios $\frac{\exp(-\theta c_2^L)}{\mathbb{E}\left[\exp(-\theta c_2^L)\right]}$ in solid lines. The distorted density used by lenders equals the likelihood ratio times the objective density.
\begin{figure}[!hbtp]\centering
    \includegraphics[width=0.95\textwidth]{distorted_noncontingent_paper.pdf}
\caption{Distorted probabilities when the government issues noncontingent debt (amount fixed at the rational-expectations eq'm level). Top: distorted densities for each $\theta$ and default probability. Bottom: likelihood ratios (distortions) and promised repayments $R(z)$
\label{Figure_distorted_noncontingent}}
\end{figure}

In the case of noncontingent debt, the expected repayment is a step function of the state $z$: the government defaults when income is low. The lenders' stochastic discount factor is therefore also a step function of the state, as marginal utility of lenders is constant (and high) to the left of the default threshold and constant (and low) to the right of it. As the robustness parameter $\theta$ increases, the metaphorical evil agent has more scope to distort probabilities, and does so by assessing the default set as more likely. For higher values of $\theta$, therefore, the expected return of the debt (under the distorted density) is lower and lenders require higher spreads to hold it.
  
Indexing debt repayments linearly to second-period output has two consequences, illustrated in Figure \ref{Figure_distorted_linear}, which is computed for an indexing coefficient $\alpha = 1$. 
\begin{figure}[!hbtp]\centering
    \includegraphics[width=0.95\textwidth]{distorted_linear_paper.pdf}
\caption{Distorted probabilities when the government issues linearly-indexed debt (amount fixed at the rational-expectations eq'm level). Top: distorted densities for each $\theta$ and default probability. Bottom: likelihood ratios (distortions) and promised repayments $R(z)$
\label{Figure_distorted_linear}}
\end{figure}
On the one hand, stipulating lower payments when output is lower successfully shrinks the ex-post default set. However, it also affects the robust lenders' probability distortions: states with higher stipulated payments are distorted downward, resulting in an overall shift to the left of the distorted distribution. When debt is indexed linearly, lenders act almost as if the output process had a lower mean. As before, the amount of distortion (but not its shape) increases with the robustness parameter $\theta$. Higher values of $\alpha$ shrink the default set but enable stronger probability distortions in the repayment set. How this tradeoff is resolved depends on the degree of robustness.
 
Finally, Figure \ref{Figure_distorted_threshold} considers the case of the threshold bond which promises to repay 1 unit of the good if the state is above its mean (under the approximating model which is shared by government, lenders, and Nature) and 0 otherwise. 
\begin{figure}[!hbtp]\centering
  \includegraphics[width=0.95\textwidth]{distorted_threshold_paper.pdf}
\caption{Distorted probabilities when the government issues the threshold bond (amount fixed at the rational-expectations eq'm level). Top: distorted densities for each $\theta$ and default probability. Bottom: likelihood ratios (distortions) and promised repayments $R(z)$
\label{Figure_distorted_threshold}}
\end{figure}
In this case, the probability distortions are much more striking. Similarly to the default threshold, the jump in stipulated repayments creates a jump in the probability distortions. But the same distortion applied to an event with higher probability results in a larger change in the density. Moreover, as we will see later, the large distortions evident in Figure \ref{Figure_distorted_threshold} manifest as high spreads that negate the gains from contingency in repayment.

\paragraph{Optimal debt design} We turn our attention to the problem of how to design state-contingent debt instruments and how the optimal design changes with the degree of robustness. When facing lenders with robustness parameter $\theta$, let $R^\star(z;\theta)$ maximize the equilibrium value attained by the government $V(\theta, R)$, subject to a non-negativity constraint.

Figure \ref{Figure_distorted_optimal_RE} is based on the case in which the repayment function $R(z)$ is optimized given that lenders have rational expectations ($\theta \to 0$).
\begin{figure}[!hbtp]\centering
    \includegraphics[width=0.95\textwidth]{distorted_optimal_RE_paper.pdf}
\caption{Distorted probabilities when the government issues debt with the optimal indexation for RE lenders (amount fixed at the rational-expectations eq'm level). Top: distorted densities for each $\theta$ and default probability. Bottom: likelihood ratios (distortions) and promised repayments $R(z)$
\label{Figure_distorted_optimal_RE}}
\end{figure}
The optimal repayment function in this case has a region where it promises zero repayments (the constraint $R\geq 0$ binds), followed by a region where repayments are increasing in the level of output. This type of debt makes the default set empty and takes advantage of contingency in repayments. Because of this, the debt designed for rational-expectations lenders is subject to large probability distortions (albeit less than the threshold debt) when evaluated by robust lenders.

Figure \ref{Figure_distorted_robust} illustrates the opposite exercise: debt designed for lenders with the highest level of robustness we consider (a value of $\theta = 4$).
\begin{figure}[!hbtp]\centering
    \includegraphics[width=0.95\textwidth]{distorted_robust_paper.pdf}
\caption{Distorted probabilities when the government issues debt with the optimal indexation for robust lenders (amount fixed at the rational-expectations eq'm level). Top: distorted densities for each $\theta$ and default probability. Bottom: likelihood ratios (distortions) and promised repayments $R(z)$
\label{Figure_distorted_robust}}
\end{figure}
In this case, the tradeoff between contingency and enabling probability distortions is resolved by a debt design which varies much less with the state. As a result, the probability distortions are much milder, at all levels of $\theta$.

Finally, Figure \ref{Figure_arrow} shows the optimal debt design $R^\star(z;\theta) = \arg\max_R V(\theta,R)$ for each value of the robustness parameter $\theta$, as well as the expected repayment of noncontingent debt (taking default into account). It is clear that, as $\theta$ increases, the optimal debt structure features less contingency, lower slopes, and an avoidance of regions with zero or low stipulated repayments.
\begin{figure}[!hbtp]\centering
    \includegraphics[width=0.9\textwidth]{arrow_paper.pdf}
\caption{Optimal design of state-contingent debt for each type of lender.
\label{Figure_arrow}}
\end{figure}
Figure \ref{Figure_arrow} sharply illustrates the tradeoffs in the debt-design problem when facing robust lenders. On the one hand, the government would like to minimize the contingency in stipulated repayments in order to prevent probability distortions. But the government also needs to minimize another source of contingency given by default risk ex-post. In low states, the government promises as much as it can credibly commit to repay.

\subsection{Spreads}\label{sec:spreads}
We now turn to how the probability distortions and concerns for model misspecification affect bond prices, issuances, and the government's welfare in equilibrium. The top panel of Figure \ref{Figure_spreads_parametric} shows our decomposition of equilibrium spreads as a function of the robustness parameter $\theta$. The bottom panel shows the issuance value $q(b)b$ as well as the welfare of the government. We measure welfare as the equivalent increase in consumption with respect to an equilibrium with the same $\theta$ but when the government issues noncontingent debt.\footnote{Somewhat abusing notation, if
\begin{align*}
  V(\theta, R, x) = u\left(c_1^b(1+x)\right) + \beta \mathbb{E}\left[u\left(c_2^b(1+x)\right)\right]
\end{align*}
is the value attained by the government, augmenting the equilibrium level of consumption by the factor $x$, then in each equilibrium with bonds $R$ and robustness $\theta$ we measure welfare by finding $x$ to make $V(\theta, R, 0) = V(\theta, 1, x)$.}
\begin{figure}[!hbtp]\centering
    \includegraphics[width=1\textwidth]{spreads_parametric_paper.pdf}
\caption{Spreads when the government issues the simple instruments. Bottom panel: issuance value $q(b)b$ and welfare difference from noncontingent debt}\label{Figure_spreads_parametric}
\end{figure}

When the government issues noncontingent debt, more robust lenders charge a higher spread for the ambiguity of default. The government responds by issuing lower amounts of debt. In our parametrization, the decrease in the default probability (the amount of risk) roughly compensates the increase in the spreads because of ambiguity (the price of risk).

Linearly-indexed debt successfully decreases the equilibrium default probability, as evidenced by lower spreads under rational-expectations. This leads to welfare gains equivalent to about $0.9\%$ of consumption from the noncontingent debt benchmark. As robustness increases, spreads from ambiguity of contingency and from ambiguity of default open up, eroding the government's ability to issue debt and therefore welfare gains. At $\theta = 4$, however, the government still values the option to move from noncontingent to linearly-indexed debt at about $0.8\%$ of consumption.

The picture is quite different for threshold debt. This type of debt completely eliminates default risk and therefore trades at no spreads with rational-expectations lenders. However, as $\theta$ increases, the large probability distortions discussed above support large spreads from the ambiguity of contingency. These high spreads quickly turn the welfare gains from state-contingent debt into welfare losses.

Figure \ref{Figure_spreads_optimal} repeats the exercise for our optimally-designed instruments.
\begin{figure}[!htbp]\centering
    \includegraphics[width=1\textwidth]{spreads_optimal_paper.pdf}
\caption{Spreads when the government issues the optimally indexed instruments. Bottom panel: issuance value $q(b)b$ and welfare difference from noncontingent debt}
\label{Figure_spreads_optimal}
\end{figure}
In the case of debt designed for rational-expectations lenders, large spreads from the ambiguity of contingency arise as lenders become more robust. Because the contingency introduced by this type of debt is less severe than the threshold debt studied above, the gains from issuing this type of state-contingent debt evaporate more slowly and remain positive at all values of $\theta$ (but would disappear for more extreme values). Finally, debt designed for robust lenders implies lower gains when priced by rational-expectations lenders, but these gains remain high as robustness increases. The effort to create little ex-ante and ex-post contingency in this type of debt keeps the ambiguity spreads at bay.

