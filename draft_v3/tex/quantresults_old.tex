\section{Results \label{sec:quantresults}}

Tables \ref{table:CE} and \ref{table:PP} report moments in the simulations of the benchmark model for each parametrization.  Following \cite{PouzoPresno2016}, we report results for pre-default simulation samples, except for the computation of default frequencies which are computed on the entire sample. We simulate the model for a number of periods that allows us to extract 1000 samples of 35 consecutive periods before a default. We focus on samples of 35 periods because we compare the artificial data generated by the model with Argentine data from the first quarter of 1993 to the last quarter of 2001.\footnote{The qualitative features of this data are also observed in other sample periods and in other emerging markets (see, for example, \citealp{AG_cycles}, \citealp{Alvarez}, \citealp{Bora}, \citealp{Neumeyer05}, and \citealp{drives}).} In order to facilitate the comparison of simulation results with the data, we only consider simulation sample paths in which the last default was declared at least four periods before the beginning
of each sample. Default frequencies are computed using all simulation data.

Table \ref{table:CE} shows the simulation results of the benchmark model (column ``Noncontingent'') under the \cite{Chatty} parametrization described above. It also shows how the simulation results change in this case if we introduce robust lenders with the value of $\theta$ calibrated by \cite{PouzoPresno2016}. Analogously, Table \ref{table:PP} shows the simulation results of the benchmark model (column Noncontingent) under the \cite{PouzoPresno2016} parametrization in Table \ref{Table_parameters}. It also shows how the simulation results change in this case if we consider rational expectations lenders as in \cite{Chatty}. Overall, the tables show that both benchmark calibrations match the data reasonably well. As in the data, in the simulations of the baseline model, consumption and income are highly correlated and the spread is countercyclical. Consumption volatility is higher than income volatility, which is consistent with the findings in \cite{Neumeyer05} and \cite{AG_cycles}. The calibrations closely match the targeted moments. The crucial difference is that only with robust lenders the model is able to match simultaneosuly the moments of the spread and default frequency in the data, which is the main contribution from \cite{PouzoPresno2016}. \cite{Chatty} is able to match the mean spread level but at the expense of a much larger default probability. \cite{PouzoPresno2016} show that the model with robust lenders is also able to match other quantiles of spread in the data.


\begin{table}[!hbtp]\centering\small 
\caption{Statistics based on \citet{Chatty}} \label{table:CE}
\begin{tabular}{@{}lccc@{}}\toprule
  & \multicolumn{3}{c}{Rational Expectations (benchmark)} \\\cmidrule{2-4}
\textbf{Statistic} & \textbf{Noncontingent} & \textbf{Threshold} & $\alpha = 1$\\\midrule
Spread                   & 8.5            & 0.6            & 6.8            \\
Std Spread               & 4.3            & 0.4            & 3.0            \\
Debt                     & 69.9           & 159.6           & 74.4           \\
Std(c)/Std(y)            & 1.24            & 0.83           & 1.21           \\
Corr(y,c)                & 0.98           & 0.53           & 0.98           \\
Corr(y,tb/y)             & -0.7          & 0.52           & -0.62          \\
Corr(y,spread)           & -0.77        & -0.87           & -0.78          \\
Default Prob             & 5.8            & 0.56            & 5.3            \\
Welfare Gains            & -              & 1.86           & 0.27           \\
  \bottomrule
\end{tabular}%
\begin{tabular}{@{}lccc@{}}\toprule
  & \multicolumn{3}{c}{$\theta = 1.6155$} \\\cmidrule{2-4}
  & \textbf{Noncontingent} & \textbf{Threshold} & $\alpha = 1$ \\\midrule
& 8.4            & 15.5           & 7.1            \\
& 4.4            & 2.3            & 3.1            \\
& 62.6           & 87.7           & 67.2           \\
& 1.25           & 0.82           & 1.22           \\
& 0.98           & 0.94           & 0.98           \\
& -0.67          & 0.58           & -0.6          \\
& -0.75          & -0.61          & -0.77           \\
& 2.3            & 0.12            & 1.8            \\
& -              & -0.87          & 0.2           \\
  \bottomrule 
\end{tabular}
\begin{tablenotes1}
  Threshold debt pays if income is above the mean and payments are linearly indexed with alpha = 1.
\end{tablenotes1}
\end{table}
 
 
  
\begin{table}[!hbtp]\centering\small
\caption{Statistics based on \citet{PouzoPresno2016}}  \label{table:PP}
\begin{tabular}{@{}lccc@{}}\toprule
  & \multicolumn{3}{c}{Rational Expectations} \\\cmidrule{2-4}
\textbf{Statistic} & \textbf{Noncontingent} & \textbf{Threshold} & $\alpha = 1$\\\midrule
Spread                   & 8.1            & 0.36            & 7.2           \\
Std Spread               & 4.5            & 0.23            & 3.7            \\
Debt                     & 48.7           & 116.5           & 50.8           \\
Std(c)/Std(y)            & 1.24           & 0.82           & 1.22           \\
Corr(y,c)                & 0.98           & 0.55           & 0.98           \\
Corr(y,tb/y)             & -0.71          & 0.54           & -0.67          \\
Corr(y,spread)           & -0.77          & -0.87           & -0.79          \\
Default Prob             & 5.5            & 0.3            & 5.3            \\
Welfare Gains            & -              & 1.19           & 0.09           \\
  \bottomrule
\end{tabular}%
\begin{tabular}{@{}lccc@{}}\toprule
& \multicolumn{3}{c}{$\theta = 1.6155$ (benchmark)} \\\cmidrule{2-4}
& \textbf{Noncontingent} & \textbf{Threshold} & $\alpha = 1$ \\\midrule
& 8.15           & 11.1           & 7.1           \\
& 4.6            & 1.58            & 3.6            \\
& 44.0           & 67.6           & 46.1           \\
& 1.25           & 0.84           & 1.23           \\
& 0.98           & 0.93           & 0.98           \\
& -0.68          & 0.52           & -0.64          \\
& -0.76          & -0.63          & -0.77          \\
& 3.0            & 0.0            & 2.6            \\
& -              & -0.37          & 0.07           \\
\bottomrule 
\end{tabular}
\begin{tablenotes1}
  Threshold debt pays if income is above the mean and payments are linearly indexed with alpha = 1.
\end{tablenotes1}
\end{table}
  
Next, we turn to the analysis of allowing the government to issue state-contingent debt. In both tables, the column ``Threshold'' shows the simulation results when the government can issue an income-indexed bond with parameters $\tau = \bar{y}$ and $\alpha = 1$. This bond structure in the model intends to capture the GDP-linked bond that Argentina issued in 2005 (the details of this bond are discussed in Section \ref{sec:2permodel}). Tables \ref{table:CE} and \ref{table:PP} show that the qualitative results from the two-period model carry over into the quantitative model. Under both parametrizations, this income-indexed bond generates substantial welfare gains if the small-open economy faces rational expectations lenders. As the government does not need to make coupon payments in any income realization lower to the mean, the default probability and, thus, the spread are almost eliminated. This allows the government to increase its indebtedness and reduce the volatility of consumption. The threshold bond effectively expands the government borrowing opportunities (larger indebtedness at more favorable prices).

However, when the government faces robust lenders, under both parametrizations the government is made worse off by the introduction of the threshold bond. In these cases, while the threshold bond still reduces the volatility of consumption relative to income and eliminates most of the default risk, it still leads to a large increase in bond spreads. The sovereign spread increases from 8.5 to 15.5 percent in the \cite{Chatty} calibration, and from 8.15 to 11.1 percent in the \cite{PouzoPresno2016} calibration. From the discussion in Section \ref{sec:2permodel}, with robust lenders the sovereign spread not only reflects the default premium but also includes the amiguity premia (from default and contingency). Given that the government never defaults with this bond structure, the default premium and the ambiguity premium associated with default are both zero. But, robust lenders' probability distortions amplify the likelihood of states in which the bond promises no repayment. This leads to a large ambiguity premium related to the contingency of the bond which explains the resulting levels of the spread and, ultimately, the associated welfare losses. Consistent with the results from the two-period model, these findings show that, through the lens of our model, the unexplained portion of the spread of the Argentinian GDP-linked bond that the literature has labeled as a novelty premium (\citealp*{ChamonCostaRicci2008}) is in fact an ambiguity premium. Moreover, the welfare losses could also rationalize why countries have not issued these bonds more frequently in practice.

Finally, we search for the optimal state-contingent bond design. For each calibration, we maximize the welfare of the sovereign by choosing the parameter values $\tau$ and $\alpha$. Table \ref{table:optimal_structure} shows the simulation results. We find that the optimal bond design depends on the type of lenders the government is facing. In particular, while the threshold level $\tau$ is similar within each calibration, the optimal degree of indexation $\alpha$ is lower when lenders feature preferences for robustness. In all cases, the optimal state-contingent bond substantialy reduces both default risk and the volatility of consumption. At the same time, this allows the government to increase its indebtedness. When lenders have rational expectations, the decline in default risk implies a similar reduction in the sovereign spread. However, when lenders are robust the sovereign spread remains around 3\% even with negligible default risk. As discussed in Section \ref{sec:2permodel}, this spread level is due to the ambiguity premium related to the contingency of the bond. Overall, choosing the optimal state-contingent bond design results in large welfare gains, although these are larger with rational expectations lenders.\footnote{We also find that for a given state-contingent structure, welfare gains are decreasing in the robustness parameter $\theta$.}

\begin{table}[!hbtp]\centering\small 
\caption{Statistics based on \citet{Chatty} and \citet{PouzoPresno2016} under the optimal state-contingent bond with and without robust lenders.} \label{table:optimal_structure}
\begin{tabular}{@{}lccc@{}}\toprule
  & \multicolumn{2}{c}{\citet{Chatty}} \\\cmidrule{2-3}
\textbf{Statistic} & \textbf{Rational Expectations} & \textbf{Robustness}\\
& $\tau$ = 0.75, $\alpha$ = 4 & $\tau$ = 0.8, $\alpha$ = 3 &\\\midrule
Spread                   &   0.02          & 2.83                        \\
Std Spread               &   0.02          & 0.11                     \\
Debt                     &    119.8        & 95.7                 \\
Std(c)/Std(y)            &     0.8        & 0.99                   \\
Corr(y,c)                &     0.99       & 0.98                   \\
Corr(y,tb/y)             &     0.98      & 0.13                  \\
Corr(y,spread)           &   -0.42         & -0.17                      \\
Default Prob             &  0.04           & 0.17                      \\
Welfare Gains            & 3.2             & 1.44                    \\
  \bottomrule
\end{tabular}%
\begin{tabular}{@{}lccc@{}}\toprule
  & \multicolumn{2}{c}{\citet{PouzoPresno2016}} \\\cmidrule{2-3}
  & \textbf{Rational Expectations} & \textbf{Robustness}  \\
& $\tau$ = 0.875, $\alpha$ = 7 & $\tau$ = 0.875, $\alpha$ = 5 &\\\midrule
&   0.1          &       2.8        \\
&   0.04          &         0.13            \\
&   79.3         &           65.9           \\
&   0.76         &           0.96          \\
&   0.99        &           0.98        \\
&   0.98        &           0.25      \\
&   -0.91        &          -0.67       \\
&  0.1             &    0.23                  \\
& 1.79              & 0.79                 \\
  \bottomrule 
\end{tabular}
\end{table}




